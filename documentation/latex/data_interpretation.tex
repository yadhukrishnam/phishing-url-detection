Based on \ref{fig:scatter1}, it could be observed that lengthier URLs and hostnames are more probabilistic  to be phishing URLs. 

\\Based on Fig. III and IV, the following observations could be made:

\begin{enumerate}

\item The length of domain names and URLs are longer for phishing URLs.
\item Majority of URLs having an IP address instead of a domain name, were identified to be phishing. 
\item Phishing websites were found to be  using URL shorteners, and did not have SSL support.
\item Most legitimate websites had $www$ in their URL.
\item The number  of phishing websites that had a top-level domain name as a subdomain or as a path was comparatively high.
\item The number of occurrences of special characters such as dots, at, question marks, ampersands, underscores, tildes, stars, colons, commas, etc were found to be more in phishing URLs rather than legitimate URLs . However, the count of hyphens were found to be more in legitimate URLs.

\end{enumerate}